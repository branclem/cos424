\section{Introduction}
\label{sec:intro}

Alcohol is a depressant drug that, when used in moderation, acts as a relaxant and social lubricant.  Responsible use of alcohol for recreation is a fundamental part of American culture and many other cultures worldwide, but heavy use can have serious consequences.  Alcohol abuse is a leading factor in violent crime and automotive fatalities, alcohol poisoning can be fatal in a short time period, and chronic alcoholism can lead to long-term liver failure and reduced brain function.

Binge drinking and overconsumption of alcohol are particularly widespread problems on college campuses, where many students are exposed to serious health risks from drinking.  Efforts targeting the elimination of alcohol consumption completely have historically failed and represent a grievous threat to individual liberties.  Instead, trends have pointed to education and healthy drinking practices to reduce the danger to students.

One thing that seems clear, though, is that not all individuals are at an equal risk of problem drinking.  If we could find some way to identify students that are most likely to engage in risky drinking behavior, alcohol education programs and other initiatives could be targeted specifically at these groups.  In this work we use survey data from the 2001 Harvard School of Public Health College Alcohol Study to show that information about a student's background, campus activities, and personal experiences and attitudes can be used to build a predictive classifier to identify students at risk of binge drinking.  By dividing the survey data into five categories of data, organized by how difficult they are to collect, we also explore how well our classifiers perform using limited information.

We begin by introducing the dataset and our methodologies for processing the data and evaluating predictive classifiers.  From there we move on to detailed explorations of the techniques we have used, including Generalized Linear Models, clustering techniques like k-means, and PCA.  Finally, we use the insights gained from using these techniques to provide an overall evaluation of our work and how it impacts the problem of binge drinking on college campuses.
